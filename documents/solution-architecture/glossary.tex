\newglossaryentry{Adobe Flash Player}
{
    name=Adobe Flash Player,
    description={Der Adobe Flash Player ist eine kostenlose Software, die multimediale Inhalte auf Webseiten, die auf Adobe Flash erstellt wurden, anzeigen lässt. Die Software ist in den letzten Jahren wegen Sicherheitslücken stark in Kritik geraten}
}

\newglossaryentry{Arena}{
    name={Arena},
    description={Eine Arena bezeichnet im Kontext von Online-Computerspielen der Raum, in dem das Spiel stattfindet}
}

\newglossaryentry{Battle Royale}{
    name=Battle Royale,
    description={Battle Royale bezeichnet ein Online-Computerspiel-Genre. Es befinden sich bis zu 100 Spieler in einem abgegrenzten Spielbereich. Dieser, wird kontinuierlich kleiner. Gewinner ist, wer am längsten überlebt}
}

\newglossaryentry{Game Server}{
    name={Game Server},
    description={Spieleserver (engl. game server) sind speziell für Mehrspieler-Spiele eingerichtete Server. Spieler können sich mit ihnen verbinden, um miteinander zu spielen. Sie verwalten die Spieldaten und synchronisieren die Handlungen der Spieler gegenseitig. Spieleserver kommen sowohl im Internet, um verschiedene Spieler weltweit zusammenzubringen, als auch lokal auf LAN-Partys, insbesondere wenn keine ausreichend schnelle Verbindung zum Internet besteht, zum Einsatz}
}

\newglossaryentry{Latenz}{
    name={Latenz},
    description={Ein sehr bekannter Begriff ist der Ping (benannt nach dem Ping-Kommando in Windows). Der Ping bezeichnet die Zeit, die ein Datenpaket von Ihrem PC zu einem Server im Internet und wieder zurück benötigt – diese Verzögerung wird auch als Latenz bezeichnet. Ist die Latenz (zu) hoch, nennt man dies auch Lag}
}

\newglossaryentry{Lobby}{
    name=Lobby,
    description={Eine Lobby bezeichnet im Kontext von Online-Computerspielen einen Warteraum. Die für eine Spielrunde angemeldeten Spieler befinden sich in diesem, bis die Runde beginnt}
}

\newglossaryentry{Multiplayer}{
    name={Multiplayer},
    description={Der Begriff Mehrspieler (englisch multiplayer, multi-player) beschreibt im Bereich der Computerspiele eine Spielart (im Computerspieler-Jargon auch Modus), bei der man mit oder gegen andere Menschen spielt. Sie ist das Gegenstück zur Einzelspieler-Spielweise, bei der ein einzelner Spieler alleine mit bzw. gegen den Computer spielt}
}

\newglossaryentry{OOP}{
    name={OOP},
    description={Die objektorientierte Programmierung (kurz OOP) ist ein auf dem Konzept der Objektorientierung basierendes Programmierparadigma. Die Grundidee besteht darin, die Architektur einer Software an den Grundstrukturen desjenigen Bereichs der Wirklichkeit auszurichten, der die gegebene Anwendung betrifft. Ein Modell dieser Strukturen wird in der Entwurfsphase aufgestellt. Es enthält Informationen über die auftretenden Objekte und deren Abstraktionen, ihre Typen. Die Umsetzung dieser Denkweise erfordert die Einführung verschiedener Konzepte, insbesondere Klassen, Vererbung, Polymorphie und spätes Binden}
}

\newglossaryentry{Unified Process (UP)}{
    name=Unified Process,
    description={UP definiert ein iterativer Entwicklungsprozess für objektorientierte Systeme}
}
