% Kein Punkt am Schluss der description notwending!
\newglossaryentry{Adobe Flash Player}
{
    name=Adobe Flash Player,
    description={Der Adobe Flash Player ist eine kostenlose Software, die multimediale Inhalte auf Webseiten, die auf Adobe Flash erstellt wurden, anzeigen lässt. Die Software ist in den letzten Jahren wegen Sicherheitslücken stark in Kritik geraten}
}

\newglossaryentry{Arena}{
    name={Arena},
    description={Eine Arena bezeichnet im Kontext von Online-Computerspielen der Raum, in dem das Spiel stattfindet}
}

\newglossaryentry{Battle Royale}{
    name=Battle Royale,
    description={Battle Royale bezeichnet ein Online-Computerspiel-Genre. Es befinden sich bis zu 100 Spieler in einem abgegrenzten Spielbereich. Dieser, wird kontinuierlich kleiner. Gewinner ist, wer am längsten überlebt}
}

\newglossaryentry{Continuous Integration}{
    name={Continuous Integration},
    description={
        Kontinuierliche Integration ist ein Begriff aus der Softwareentwicklung. Es beschreibt das Konzept des fortlaufenden Zusammenführens von Komponenten einer Anwendung. Dabei werden üblicherweise automatische Tests durchgeführt, um die Qualität der Software nach jedem Integrationsschritt zu gewährleisten.
    }
}

\newglossaryentry{Game Server}{
    name={Game Server},
    description={Spieleserver (engl. game server) sind speziell für Mehrspieler-Spiele eingerichtete Server. Spieler können sich mit ihnen verbinden, um miteinander zu spielen. Sie verwalten die Spieldaten und synchronisieren die Handlungen der Spieler gegenseitig. Spieleserver kommen sowohl im Internet, um verschiedene Spieler weltweit zusammenzubringen, als auch lokal auf LAN-Partys, insbesondere wenn keine ausreichend schnelle Verbindung zum Internet besteht, zum Einsatz}
}

\newglossaryentry{HTML5 Canvas}{
    name={HTML5 Canvas},
    description={Das Canvas Element ist ein HTML5-Element, dass als Kontainer benutzt wird, um mit Hilfe von Javascript auf einer Webseite Grafiken, wie Simulationen zu zeichnen}
}

\newglossaryentry{JSP}{
    name=JSP,
    description={JavaServer Pages (kurz JSP) beschreibt ein Java-Framework um mit Java serverseitig dynamisch XML-Seiten zu kreiiren, das auf dem Konzept der Java-Servlets basiert.}
}

\newglossaryentry{Latenz}{
    name={Latenz},
    description={Ein sehr bekannter Begriff ist der Ping (benannt nach dem Ping-Kommando in Windows). Der Ping bezeichnet die Zeit, die ein Datenpaket von Ihrem PC zu einem Server im Internet und wieder zurück benötigt – diese Verzögerung wird auch als Latenz bezeichnet. Ist die Latenz (zu) hoch, nennt man dies auch Lag}
}

\newglossaryentry{Lobby}{
    name=Lobby,
    description={Eine Lobby bezeichnet im Kontext von Online-Computerspielen einen Warteraum. Die für eine Spielrunde angemeldeten Spieler befinden sich in diesem, bis die Runde beginnt},
    plural={Lobbys}
}

\newglossaryentry{Multiplayer}{
    name={Multiplayer},
    description={Der Begriff Mehrspieler (englisch multiplayer, multi-player) beschreibt im Bereich der Computerspiele eine Spielart (im Computerspieler-Jargon auch Modus), bei der man mit oder gegen andere Menschen spielt. Sie ist das Gegenstück zur Einzelspieler-Spielweise, bei der ein einzelner Spieler alleine mit bzw. gegen den Computer spielt}
}

\newglossaryentry{OOP}{
    name={OOP},
    description={Die objektorientierte Programmierung (kurz OOP) ist ein auf dem Konzept der Objektorientierung basierendes Programmierparadigma. Die Grundidee besteht darin, die Architektur einer Software an den Grundstrukturen desjenigen Bereichs der Wirklichkeit auszurichten, der die gegebene Anwendung betrifft. Ein Modell dieser Strukturen wird in der Entwurfsphase aufgestellt. Es enthält Informationen über die auftretenden Objekte und deren Abstraktionen, ihre Typen. Die Umsetzung dieser Denkweise erfordert die Einführung verschiedener Konzepte, insbesondere Klassen, Vererbung, Polymorphie und spätes Binden}
}

\newglossaryentry{Reverse-Proxy}{
    name=Reverse-Proxy,
    description={Beantwortet die Anfragen eines Clients durch Anfragen von Servern im internen Netzwerk. Diese müssen für den Client nicht sichtbar sein}
}

\newglossaryentry{SPA}{
    name={SPA},
    description={Eine Single Page Application (kurz SPA) nennt man eine Webseite, die ihren Inhalt dynamisch vom Server lädt und einbaut. Somit wird die Seite bei einer Veränderung des Inhalts nicht gesammthaft neu geladen}
}

\newglossaryentry{Unified Process (UP)}{
    name=Unified Process,
    description={UP definiert ein iterativer Entwicklungsprozess für objektorientierte Systeme}
}

\newglossaryentry{Webserver}{
    name=Webserver,
    description={Ein Webserver ist ein Server, der Informationen an Clients, wie z.B. Webbrowsers überträgt}
}

\newglossaryentry{Websocket}{
    name=Websocket,
    description={Das WebSocket-Protokoll ist ein auf TCP basierendes Netzwerkprotokoll, das entworfen wurde, um eine bidirektionale Verbindung zwischen einer Webanwendung und einem WebSocket-Server bzw. einem Webserver, der auch WebSockets unterstützt, herzustellen},
    plural={Websockets}
}

\newglossaryentry{React}{
    name={React},
    description={React ist eine JavaScript-Softwarebibliothek, die ein Grundgerüst für die Ausgabe von User-Interface-Komponenten von Webseiten zur Verfügung stellt (Webframework). Komponenten werden in React hierarchisch aufgebaut und können in dessen Syntax als selbst definierte HTML-Tags repräsentiert werden}
}

\newglossaryentry{Gastaccount}
{
    name=Gastaccount,
    description={Ein Gastaccount ist ein anonymer Account. Dieser dient dem Spieler eine schnelle Einstiegsmöglichkeit ins Spiel, ohne sich anzumelden. Während der Spieler den Gastaccount benutzt werden seine Punkte nicht in der Datenbank gespeichert}
}

\newglossaryentry{Webbrowser}
{
    name=Webbrowser,
    description={Webbrowser oder allgemein auch Browser (engl., to browse, ‚stöbern, schmökern, umsehen‘, auch ‚abgrasen‘) sind spezielle Computerprogramme zur Darstellung von Webseiten im World Wide Web oder allgemein von Dokumenten und Daten. Das Durchstöbern des World Wide Webs beziehungsweise das aufeinanderfolgende Abrufen beliebiger Hyperlinks als Verbindung zwischen Webseiten mit Hilfe solch eines Programms wird auch als Internetsurfen bezeichnet}
}

\newglossaryentry{HTTPS}
{
    name=HTTPS,
    description={Hypertext Transfer Protocol Secure (HTTPS, englisch für „sicheres Hypertext-Übertragungsprotokoll“) ist ein Kommunikationsprotokoll im World Wide Web, mit dem Daten abhörsicher übertragen werden können. Es stellt eine Transportverschlüsselung dar}
}

% Kein Punkt am Schluss der description notwending!